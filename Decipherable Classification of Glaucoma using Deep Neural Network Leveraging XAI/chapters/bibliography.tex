%\section{Bibliography}
\justifying
\section*{Bibliography} 
\noindent[1] Salam, A. A., Khalil, T., Akram, M. U., Jameel, A., & Basit, I. (2016). Automated detection of glaucoma using structural and non structural features. SpringerPlus, 5(1).\\ https://doi.org/10.1186/s40064-016-3175-4

\vspace{5mm}
\noindent[2] Ran, A., & Cheung, C. Y. (2021). Deep Learning-Based Optical Coherence Tomography and Optical Coherence Tomography Angiography Image Analysis: An Updated Summary. Asia-Pacific Journal of Ophthalmology, 10(3), 253–260. 

\noindent https://doi.org/10.1097/apo.0000000000000405

\vspace{5mm}
\noindent[3] Aleci, C. (2020). Detection of Visual Field Loss Progression in Glaucoma: An Overview and Food for Thought. Ophthalmology Research: An International Journal, 16–24.\\
https://doi.org/10.9734/or/2020/v13i130158

\vspace{5mm}
\noindent[4] Saha, S., Wang, Z., Sadda, S., Kanagasingam, Y., & Hu, Z. (2020). Visualizing and understanding inherent features in SD‐OCT for the progression of age‐related macular degeneration using deconvolutional neural networks. Applied AI Letters, 1(1). 

\noindent https://doi.org/10.1002/ail2.16
 
\vspace{5mm}
\noindent[5] Civit-Masot, J., Dominguez-Morales, M. J., Vicente-Diaz, S., & Civit, A. (2020). Dual Machine-Learning System to Aid Glaucoma Diagnosis Using Disc and Cup Feature Extraction. IEEE Access, 8, 127519–127529. 

\noindent https://doi.org/10.1109/access.2020.3008539

\vspace{5mm}
\noindent[6] Diabetic Retinopathy | National Eye Institute. (2021, July 30). National Eye Institute. 

\noindent https://www.nei.nih.gov/learn-about-eye-health/eye-conditions-and-diseases/diabetic-retinopathy

\vspace{5mm}
\noindent[7] Types of Glaucoma.(2020, June 2). Glaucoma Research Foundation. 

\noindent https://www.glaucoma.org/glaucoma/types-of-glaucoma.php#:\%7E:text=There\%20are\% 20several\%20types\%20of,or\%20pressure\%20inside\%20the\%20eye

\vspace{5mm}
\noindent[8] Saba, T., Khan, M. W., Yasmin, M., & Sharif, M. (2017). CDR based glaucoma detection using fundus images: a review. International Journal of Applied Pattern Recognition, 4(3), 261. 

\noindent https://doi.org/10.1504/ijapr.2017.10007613

\vspace{5mm}
\noindent[9] Thakoor, K. A., Li, X., Tsamis, E., Sajda, P., & Hood, D. C. (2019). Enhancing the Accuracy of Glaucoma Detection from OCT Probability Maps using Convolutional Neural Networks. 2019 41st Annual International Conference of the IEEE Engineering in Medicine and Biology Society (EMBC). 

\noindent https://doi.org/10.1109/embc.2019.8856899

\vspace{5mm}
\noindent[10]  Lim, T. C., Chattopadhyay, S., & Acharya, U. R. (2012). A survey and comparative study on the instruments for glaucoma detection. Medical Engineering & Physics, 34(2), 129–139. 

\noindent https://doi.org/10.1016/j.medengphy.2011.07.030

\vspace{5mm}
\noindent[11] Muddamsetty, S. M., Jahromi, M. N. S., & Moeslund, T. B. (2021b). Expert Level Evaluations for Explainable AI (XAI) Methods in the Medical Domain. Pattern Recognition. ICPR International Workshops and Challenges, 35–46. 

\noindent https://doi.org/10.1007/978-3-030-68796-0_3

\vspace{5mm}
\noindent[12] Abbas, Q. (2017). Glaucoma-Deep: Detection of Glaucoma Eye Disease on Retinal Fundus Images using Deep Learning. International Journal of Advanced Computer Science and Applications, 8(6). 

\noindent https://doi.org/10.14569/ijacsa.2017.080606

\vspace{5mm}
\noindent[13]  Dervisevic, E., Pavljasevic, S., Dervisevic, A., & Kasumovic, A. (2016). Challenges In Early Glaucoma Detection. Medical Archives, 70(3), 203. 

\noindent https://doi.org/10.5455/medarh.2016.70.203-207

\vspace{5mm}
\noindent[14]  R. Asaoka, H. Murata, A. Iwase, M. Araie, “Detecting preperimetric Glaucoma with Standard Automated Perimetry Using a Deep Learning Classifier,” Ophthalmology, vol. 123, pp. 1974–1980, September 2016.

\vspace{5mm}
\noindent[15]  Mayro, E.L., Wang, M., Elze, T. et al. The impact of artificial intelligence in the diagnosis and management of glaucoma. Eye 34, 1–11 (2020). 

\noindent https://doi.org/10.1016/j.ophtha.2016.05.029

\vspace{5mm}
\noindent[16]  Shabbir, A., Rasheed, A., Shehraz, H., Saleem, A., Zafar, B., Sajid, M., Ali, N., Dar, S. H., & Shehryar, T. (2021). Detection of glaucoma using retinal fundus images: A comprehensive review. Mathematical Biosciences and Engineering, 18(3), 2033–2076. 

\noindent https://doi.org/10.3934/mbe.2021106

\vspace{5mm}
\noindent[17] Brown, J. M., & Leontidis, G. (2021). Deep learning for computer-aided diagnosis in ophthalmology: a review. State of the Art in Neural Networks and their Applications, 219-237.


\noindent https://www.sciencedirect.com/science/article/pii/B9780128197400000115

\vspace{5mm}
\noindent[18] Sau, P. C., Gupta, M., & Kumar, D. (2021). A Comparative Study: Glaucoma Detection Using Deep Neural Networks. In Proceedings of International Conference on Big Data, Machine Learning and their Applications (pp. 85-97). Springer, Singapore.

\noindent https://link.springer.com/chapter/10.1007/978-981-15-8377-3_8

\vspace{5mm}
\noindent[19] Greco, A., Rizzo, M. I., De Virgilio, A., Gallo, A., Fusconi, M., de Vincentiis, M. (2016). Emerging concepts in Glaucoma and review of the literature. The American Journal of Medicine (2016).

\noindent https://doi.org/10.1016/j.amjmed.2016.03.038

\vspace{5mm}
\noindent[20] Pascolini, D., & Mariotti, S. P. (2012). Global estimates of visual impairment: 2010. British Journal of Ophthalmology, 96(5), 614-618.

\noindent https://bjo.bmj.com/content/96/5/614.short

\vspace{5mm}
\noindent[21] Goldbaum MH, Sample PA, White H, Colt B, Raphaelian P, Fechtner RD, et al. Interpretation of automated perimetry for glaucoma by neural network. Invest Ophthalmol Vis Sci. 1994;35:3362–73. 


\noindent\url{https://www.researchgate.net/publication/15141759_Interpretation_of_automated\\_perimetry_for_glaucoma_by_neural_network}

\vspace{5mm}
\noindent[22]  Murphy, A. M., & Moore, C. M. M. (2020). Fully connected neural network. 

\noindent https://radiopaedia.org/articles/fully-connected-neural-network

\vspace{5mm}
\noindent[23] Brownlee, J. (2020, August 14). What is Deep Learning? Machine Learning Mastery. 

\noindent https://machinelearningmastery.com/what-is-deep-learning/

\vspace{5mm}
\noindent[24] Khan, S. M. K. (2021, January 3). Papers with Code - LAG Dataset. The Lancet. 

\noindent https://paperswithcode.com/dataset/lag

\vspace{5mm}
\noindent[25] How to fine-tune your artificial intelligence algorithms 

\noindent https://www.allerin.com/blog/how-to-fine-tune-your-artificial-intelligence-algorithms

\vspace{5mm}
\noindent[26] Saxena, A., Vyas, A., Parashar, L., & Singh, U. (2020, July). A glaucoma detection using a convolutional neural network. In 2020 International Conference on Electronics and Sustainable Communication Systems (ICESC) (pp. 815-820). IEEE.

\noindent https://ieeexplore.ieee.org/abstract/document/9155930/

\vspace{5mm}
\noindent[27] Dervisevic, E., Pavljasevic, S., Dervisevic, A., & Kasumovic, S. S. (2016). Challenges In Early Glaucoma Detection. Medical archives (Sarajevo, Bosnia and Herzegovina), 70(3), 203–207. 

\noindent https://doi.org/10.5455/medarh.2016.70.203-207

\vspace{5mm}
\noindent[28] Mash, Robert & Becherer, Nicholas & Woolley, Brian & Pecarina, John. (2016). Toward aircraft recognition with convolutional neural networks. 

\noindent https://doi.org/10.1109/NAECON.2016.7856803.

\vspace{5mm}
\noindent[29] Guo, T., Dong, J., Li, H., & Gao, Y. (2017). Simple convolutional neural network on image classification. 2017 IEEE 2nd International Conference on Big Data Analysis (ICBDA). 

\noindent https://doi.org/10.1109/icbda.2017.8078730

\vspace{5mm}
\noindent[30] Yamashita, R., Nishio, M., Do, R. K. G., & Togashi, K. (2018). Convolutional neural networks: an overview and application in radiology. Insights into Imaging, 9(4), 611–629. 

\noindent https://doi.org/10.1007/s13244-018-0639-9

\vspace{5mm}
\noindent[31] Yun-Cheng Tsai Predict Forex Trend via Convolutional Neural Networks 

\noindent https://www.degruyter.com/document/doi/10.1515/jisys-2018-0074/html

\vspace{5mm}
\noindent[32] Kaushik, A. (2020, February 26). Understanding the VGG19 Architecture. OpenGenus IQ: Computing Expertise & Legacy. 

\noindent\url{https://iq.opengenus.org/vgg19-architecture/#:\%7E:text=VGG19\%20is\%20a\\\%20variant\%20of,VGG19\%20has\%2019.6\%20billion\%20FLOPs.}

\vspace{5mm}
\noindent[33] Huang, G. (2016, August 25). Densely Connected Convolutional Networks. ArXiv.Org. 

\noindent https://arxiv.org/abs/1608.06993

\vspace{5mm}
\noindent[34] S. Saha, A comprehensive guide to convolutional neural networks — the eli5
way, Available at 

\noindent https://towardsdatascience.com/a-comprehensive-guide-toconvolutional-neural-networks-the-eli5-way-3bd2b1164a53, 2018. 

\vspace{5mm}
\noindent [35] Jeong, J. (2021, December 7). The Most Intuitive and Easiest Guide for Convolutional Neural Network. Medium. \\
\enablehyph{https://towardsdatascience.com/the-most-intuitive-and-easiest-guide-for-convolutional-neural-network-3607be47480}



\vspace{5mm}
\noindent[36] Singh, S. P. (2019, March 2). Fully Connected Layer: The brute force layer of a Machine Learning model. OpenGenus IQ: Computing Expertise & Legacy. 

\noindent https://iq.opengenus.org/fully-connected-layer/

\vspace{5mm}
\noindent[37] Sergey Ioffe, Christian Szegedy.(2015). Batch Normalization: Accelerating Deep Network Training b y Reducing Internal Covariate Shift. 

\noindent https://arxiv.org/pdf/1502.03167.pdf
 
\vspace{5mm} 
\noindent[38] S. Wager, S. Wang, and P. S. Liang, training as adaptive regularization”, in Advances in Neural Information Processing Systems 26, C. J. C. Burges, L. Bottou, M. Welling, Z. Ghahramani, and K. Q. Weinberger, Eds., Curran Associates, Inc., 2013, pp. 351-359. [Online].  

\noindent http://papers. nips.cc/paper/4882-dropout-training-as-adaptive-regularization.pdf.

\vspace{5mm}
\noindent[39] Springenberg, J. T. (2014, December 21). Striving for Simplicity: The All Convolutional Net. ArXiv.Org. https://arxiv.org/abs/1412.6806

\vspace{5mm}
\noindent[40] Dağlarli, E. (2020). Explainable Artificial Intelligence (XAI) Approaches and Deep Meta-Learning Models. Advances and Applications in Deep Learning, 79. 

\noindent https://www.intechopen.com/chapters/72398

\vspace{5mm}
\noindent[41] Shabbir, A., Rasheed, A., Shehraz, H., Saleem, A., Zafar, B., Sajid, M., ... & Shehryar, T. (2021). Detection of glaucoma using retinal fundus images: A comprehensive review. Mathematical Biosciences and Engineering, 18(3), 2033-2076. 

\noindent http://aimspress.com/article/doi/10.3934/mbe.2021106

\vspace{5mm}
\noindent[42] Vejjanugraha, P., Kongprawechnon, W., Kondo, T., Tungpimolrut, K., & Kotani, K. (2017). An automatic screening method for primary open-angle glaucoma assessment using binary and multi-class support vector machines. ScienceAsia, 43(4), 229. 

\noindent https://doi.org/10.2306/scienceasia1513-1874.2017.43.229