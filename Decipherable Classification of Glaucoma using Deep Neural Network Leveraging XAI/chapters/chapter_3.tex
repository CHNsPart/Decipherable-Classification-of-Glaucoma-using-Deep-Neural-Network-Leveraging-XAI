%\section{Literature Review}
\section{Literature Review} 
One of the most prevalent causes of permanent blindness is glaucoma, around the world, from the article [10]. As when the pressure inside the eye is too high in a particular nerve that moment glaucoma will develop and it will also create eye ache. The working mechanisms of the different diagnosis tools like tonometers, gonioscopy, scanning laser tomography, etc. are available for the treatment and detection but there are some advantages and disadvantages which sometimes create boundaries. For this, there should be an evaluation of how this works. But with using deep learning the boundaries can be removed. As the XAI concept can be understood by humans which will be closer to the human brain to understand.

\vspace{5mm}
\noindent Recent breakthroughs in machine learning (ML) have the potential to significantly enhance retinal disease screening and diagnostic accuracy according to [11]. The recent most demanding field in XAI attempts to focus for Glaucoma disease detection. As a result, the necessity for expert-level review in assessing their efficacy becomes unavoidable. In a series of tests, we illustrate the efficacy of our method. XAI is vastly classified in these categories- Application Grounded Evaluation, Human-Grounded Evaluation, and Functionally Grounded Evaluation. Some of the findings from this paper are as follows: (i) Recent breakthroughs in machine learning promise to significantly enhance retinal disease screening and diagnostic accuracy. Multiple eye illnesses, such as diabetic retinopathy, age-related macular degeneration (AMD), glaucoma, and other abnormalities linked with retinal diseases, are being tracked. This has been diagnosed with expert-level accuracy using systems built using these methodologies. (ii) The goal of XAI is to decode Artificial Intelligence decisions, such as Deep Learning as well as Machine Learning black boxes, to the point where they are human-interpretable. (iii) Used two of the most current visual explanation ways for evaluating the clear illustration upon this given dataset which are SIDU and GRAD-CAM. As a result, in addition to enhancing the tool’s accuracy, the concept of trust, as well as the requirement for openness and robustness, emphasizes the need of investigating the impact of expert review in the context of XAI approaches.

\vspace{5mm}
\noindent Computer-aided diagnostics(CADx) tools are still struggling to detect glaucoma eye illness according to author Qaisar Abbas [12]. Glaucoma disease is the leading cause of vision impairment worldwide. His writings revealed that the Softmax linear classifier makes the ultimate judgment to distinguish differentiate between glaucoma and non-glaucoma pictures of the retinal fundus.  In glaucoma-deep, the the recommended strategy was tested on twelve hundred retinal images gathered from publicly and the datasets that are available privately. The performance of the Glaucoma-Deep system was then assessed using statistical metrics of sensitivity (SE), specificity (SP), precision (PRC) and accuracy (ACC). In general, this resulted in SE values of 84.50 percent, SP values of 98.01 percent, ACC values of 99 percent, and PRC values of 84 percent. When compared to other systems, the Nodular-Deep approach produced much better outcomes. As a result, in the Glaucoma-Deep system can quickly identify glaucoma eye illness, solving the issues in clinical specialists in the time of large-scale eye-screening processes.

\vspace{5mm}
\noindent The disease Glaucoma is the world's biggest cause of blindness, as there is no cure. [13]. It can easily lead to permanent blindness if not detected early. If eyesight loss can be detected early enough, there are treatments available to prevent it. Because it is a significant chronic eye condition that leads to irreversible blindness. Glaucoma has been on the rise in recent years. Faults in the nerve fiber layer of the retina are diagnosed before apparent abnormalities at the optic nerve's head and defects in visual field when 40 percent of axons have been irreversibly destroyed. The World Health Organization (WHO) claims that and also the World Association of Glaucomatologists (WGO), 66.8 million individuals worldwide suffered from glaucoma in 2010, with 6.7 million becoming blind as a result of the disease.

\vspace{5mm}
\noindent [6] created a new version of deep-learning (DL) algorithm to diagnose glaucoma illness by extracting numerous metrics such as fifty-two total deviations, mean deviation, and pattern standard deviation values. A Deep learning classification model, such as a deep feed-forward neural network (FNN), were utilized by the authors in this case. In contrast, the authors combined their Deep Learning predictor with earlier machine learning classifiers such as random forests as RF, gradient boosting, support vector machine, and neural network as NN. As a consequence, the scientists developed a deep ensemble solution for detecting glaucoma disease. A deep FNN classifier was used to get 92.5 percent of the AUC value, according to the authors.

\vspace{5mm}
\noindent We learned that artificial intelligence (AI) influence on glaucoma diagnosis and monitoring from [15]. Field of vision screening has now become a basis in diagnosing and controlling glaucoma because to computerized automated visual field testing, which represents a major advance in mapping the island of vision. In 1994, Goldbaum et al.[8] created a two-layer neural network for evaluating visual fields. This network has the same sensitivity (65\%) and validity (72\%) as 2 different glaucoma experts in classifying normal and glaucomatous eyes.

\vspace{5mm}
\noindent According to the writer [16] To diagnose illnesses, different healthcare systems employ content-based picture analysis and computer vision algorithms. Fundus pictures recorded with a fundus camera are used to identify abnormalities in the human eye. Amongst all, glaucoma disease has become the most common reason for neurodegenerative sickness among eye illnesses. Glaucoma has no symptoms in its early stages, and if the condition is not treated, it can result in total blindness. Glaucoma can be detected early enough to prevent irreversible visual loss. Although manual inspection of the human eye has become a viable option, although this is reliant on peoples’ effort. So, main goal in this review article is to provide a complete overview of the numerous varieties of glaucoma, their causes, prospective treatments, publicly accessible image benchmarks, performance measures, and various digital image processing, computer vision, and deep learning approaches. This research paper examines a variety of research models which was published for detecting glaucoma, ranging from low-level feature extraction to contemporary deep learning developments. The advantages and disadvantages of each strategy are examined in-depth, and the findings of each category are summarized using tabular representations.

\vspace{5mm}
\noindent As previous data shows how glaucoma disease gradually leads to blindness [8]. If we can detect glaucoma early it can be preventable against developing more serious conditions, they claimed. CDR or Cup-to-disc Ratio is a very important clinical indicator for glaucoma diagnosis in their research. Their objective is to develop a system which will give a proper path to measure Cup-to-disc Ratio results with the highest possible accuracy. Researchers evaluated the performance using 44 retinal photographs from Mettapracharak Medical, 29 of those were of patients without glaucoma and 15 of which were of patients having glaucoma condition. Which shows impressive accuracy. Next, if the CDR result is more than 0.65, the patient is considered a probable glaucoma condition. With a power raw transformation technique, CDR value between clinical outcome and edge detection is calculated. The recommended strategy yielded 5.14 percent. For optic disc segmentation, optic cup segmentation, and CDR, the percentage error utilizing their suggested technique is 2.49 percent, 5.8 percent, and 5.14 percent, accordingly. Here, the CDR is a crucial clinical sign for determining a person’s risk of developing glaucoma. For this, in their paper, the authors provided a method for autonomously calculating the CDR from fundus photos, according to the author.

\vspace{5mm}
\noindent Following the enormous success of one class of mathematical models, the artificial neural network, artificial intelligence, or AI has risen. Deep learning, a recently invented method, has taken over current scientific discourse, penetrating areas such as physics, chemistry, engineering, biology, and medicine [17]. This leads to a discussion on current solutions and state-of-the-art, with some drawbacks that may limit clinical adoption. Glaucoma is usually linked with increased intraocular pressure (IOP), which affects the overall visual field of the eye over time [18][19]. Research on glaucoma suggests that the disease’s development is influenced by several interconnected bodily mechanisms. In two types of glaucoma the angle refers to the length of contact between the iris and the cornea; if the length is long, the related illness is called open-angle glaucoma; if the length is short, it is called closed-angle glaucoma, they added [19]. Not only can glaucoma affect the patient’s eyesight, but it’s also linked to a hearing disability [19]. They reviewed the complete techniques of detecting Glaucoma affected eyes using Deep neural networks.

\vspace{5mm}
\noindent The pathogenesis of glaucoma appears to be dependent in various interconnected pathogenetic techniques, including mechanical effects characterized by excessive intraocular pressure, reduced neutrophil produce, Hypoxia, excitotoxicity, oxidative stress, and autoimmune mechanisms all play a role according to new evidence [19]. Hearing loss has also been linked to the development of glaucoma. Antiphosphatidylserine antibodies of the immunoglobulin G class were found to be more frequent in normal-tension glaucoma patients having deafness than in normal-tension glaucoma patients having normacusis. Glaucoma impacts around sixty million people globally, according to the WHO. Glaucoma is estimated to affect roughly 80 million individuals by 2020, resulting in 11.2 million cases of bilateral blindness [20]. For this reason, it needs to be treated as early as possible according to the authors.

\vspace{5mm}
\noindent The visual fields from an automated perimeter were taught to be interpreted by neural networks. The scientists tested the trained neural networks’ capacity to distinguish between normal and glaucoma-affected eyes [21]. After research, we got that, Glaucoma specialists and a trained two-layered network both got around 67 percent of the answers right. The two glaucoma experts had a sensitivity of 59 percent, while the two-layered network had a sensitivity of 65 percent. For the specialists and the two-layered network, the corresponding specificities were 74\% and 71\%, respectively. About 74 percent of the time, the experts and the network agreed, indicating that there was no substantial discrepancy between the testing methodologies. The most relevant visual field locations were discovered using feature analysis and a one-layered network. Here the authors conclude that a neural network may be taught to evaluate visual fields for glaucoma as well as a professional reader. The researchers compared the backpropagation learning approach used by automated neural networks to the methods employed by two glaucoma specialists to identify the center’s 24 degrees automated perimetry visual fields from 60 normal and 60 glaucomatous eyes. However, the neural network like deep learning has many limitations like a vast number of pictures must be incorporated in DL algorithms for them to predict with high sensitivity and specificity, moreover obtaining and storing a large number of photos comes with time limits and technological challenges. Furthermore, for such databases to remain current and prevent system-wide algorithm failure, they may need to be updated regularly. And most importantly Because the mechanism of DL’s prediction is unclear, it is referred to as a” black box”, which clearly shows its limitations. To resolve these issues using the XAI can be a big step towards Glaucoma detection.

\nomenclature{$OCT$}{Optical Coherence Tomography}
\nomenclature{$AMD$}{Age-related Macular Degeneration}
\nomenclature{$CADx$}{Computer-aided diagnostics}