%\section{Introduction}
\section{Motivation} 
Glaucoma is one of the most painful diseases caused by excessive levels of pressure in the eyes which creates a permanent loss of vision. It is also known as the ‘silent thief of sight’ as it cannot be detected at an early stage [1]. Around 57.5 million people worldwide are affected with Glaucoma. We are going to use Explainable AI (XAI) to classify scanned images of eyes that have glaucoma. XAI proposes the report to the decision of Artificial Intelligence which means Deep Learning or to the extent that it can be interpreted by humans, it's a black box. Artificial Intelligence (AI) has a subset called Deep Learning (DL) dependent on profound neural networks which have made striking leaps forwards in clinical imaging, especially for image characterization and pattern acknowledgment [2]. The use of Deep Learning (DL) is increasing in glaucoma research because these models can accomplish high precision, issues with trust, interpretability, and experimental utility structure hindrances to occurring clinical practice. The main goal of this research is to represent whether and how deep learning-based measurements can be utilized for glaucoma execution in the clinic [3]. Unfortunately, not many people bother about the early detection of Glaucoma whereas it can be diagnosed early to prevent eyesight loss. For this reason, we have decided to work with early detection of glaucoma disease to contribute to society for the good.

\section{Introduction}
Glaucoma is an eye illness wherein the optic nerve is harmed prompting irreversible loss of vision because of excessive pressure in eyes. The ciliary body emits fluid humor into the posterior chamber, which is the gap between the iris and the focus point, which is created by the eye. It then travels through the pupil and into the first chamber between the iris and cornea. It then leaves the eye after passing through a wipe-like design at the iris's base, known as the trabecular meshwork. In a solid eye, the rate at which discharge adjusts the rate of seepage. In individuals with glaucoma, the seepage waterway is part of the way or totally hindered. Liquid develops in the chambers and this expands pressure inside the eye. The pressing factor drives the focal point back and pushes on the glassy body which thus packs and harms the veins and nerve strands running at the rear of the eye. These harms bring about patches of vision misfortune, and whenever left untreated, may prompt absolute visual impairment. Glaucoma is divided into two types: First one is open-angle and second one is angle-closure.

\vspace{5mm}
\noindent Open-angle glaucoma is brought about by an incomplete blockage of the waste trench. The point between the cornea and the iris is- open, which means the passageway to the waterway is clear, however, the progression of watery humor is more slow than typical. The pressing factor develops continuously in the eye throughout a significant period. Side effects show up continuously, beginning from fringe vision misfortune, and may go on unseen until focal vision is influenced. Movement of glaucoma can be halted with medicines, however, part of the vision that is now lost can’t be reestablished. This is the reason it’s vital to distinguish early indications of glaucoma with standard eye tests.

\vspace{5mm}
\noindent Angle-closure glaucoma brought about by an abrupt and complete blockage of fluid humor waste. The pressing factor inside the eye rises quickly and may prompt absolute visual deficiency rapidly. Certain anatomical highlights of the eye, for example, limited seepage point, shallow foremost chamber, slender and saggy iris, make it simpler to foster intense glaucoma. Commonly, this happens when the understudy is expanded and the focal point has adhered to the rear of the iris. This squares the fluid humor from coursing through the understudy into the foremost chamber. Collection of liquid in the back chamber pushes on the iris, making it swell outward and square the waste point. Acute angle-closure glaucoma is a visual crisis and requires quick consideration through early detection.

\vspace{5mm}
\noindent Glaucoma, on the other hand, is one of the primary causes of blindness in people over 60. Statistics show that even with the treatment 15\% to 20\% of patients become blind. For this reason, in this research, we are going to apply Explainable AI to detect Glaucoma in a better way than exists. As the diagnosis of glaucoma is a complicated and expensive process, the application of a Deep Neural network leveraging XAI can give more improvement in understanding or detecting many problems related to glaucoma disease.


\section{Research Objectives}
The field of Explainable Artificial Intelligence has filled dramatically lately with innovations, techniques, and applications arising at a fast rate. A considerable lot of these progressions have been utilized to improve the conclusion and the executives of glaucoma. We intend to give an outline of ongoing distributions in regards to the utilization of man-made consciousness to improve the recognition and treatment of glaucoma.

\vspace{5mm}
\noindent According to modern medical science, glaucoma is diagnosed in four different ways. Initially, glaucoma is diagnosed by machine where the pressure is measured inside the eye. Here the whole diagnostic test is called Tonometry and the intraocular pressure is measured throughout this process. Apart from this, there is a prerequisite of this test which is a visual feel check. In this investigation, the doctor will move his hand from top to bottom, bottom to top, left to right, and right to left while the patient will close their eyes during the test. In this way, the doctor checks the patient’s visibility through one eye.

\vspace{5mm}
\noindent Moreover, there is another way of diagnosing glaucoma which is the Imaging Test where the main motive is to check the depth of eyes by showing pictures. However, the final diagnosis is pachymetry where The definitive diagnosis, however, is pachymetry, which is a medical gadget that measures the thickness of the cornea of the eye. In the above ways, glaucoma can be diagnosed and partial or complete blindness could be prevented.

\vspace{5mm}
\noindent AI classifiers and deep learning algorithms have been created to self-sufficiently recognize early primary and useful changes of glaucoma utilizing diverse imaging and testing modalities like fundus photography, optical cognizance tomography, and standard computerized perimetry. Artificial Intelligence has additionally been utilized to further portray structure-work connection in glaucoma. Additionally, “structure-structure” predictions have been effectively assessed. Other AI strategies using complex measurable demonstrating have been utilized to distinguish glaucoma movement, just as to foresee future movement. Though not yet endorsed for clinical use, these artificial intelligence methods can essentially improve glaucoma analysis and the board.

\vspace{5mm}
\noindent Therefore, in this research, we have focused on our models and dataset to make it efficient to predict glaucoma at an early stage. Meanwhile the models will be used to calculate the accuracy for comparison purposes. The significant contributions of this thesis are stated as follows: Our model ResNet50 has achieved the highest score among the other models with a validation accuracy of 94.7\%. Moreover, VGG-19 and ResNet50 were the Good-Fit than the other models. So the rest of the report has been organized in the following manner where we explain the proposed model, then the implementation process with results and discussion, conclusion and bibliography to bring a conclusion in our overall research paper.


\nomenclature{$XAI$}{Explainable AI}
\nomenclature{$DL$}{Deep Learning}
\nomenclature{$AI$}{Artificial Intelligence}
\nomenclature{$CNN$}{Convolution Neural Network}
\nomenclature{$SVM$}{Support Vector Machine}
\nomenclature{$FCNN$}{Fully Connected Neural Network}
\nomenclature{$LIME$}{Local Interpretable Model-agnostic Explanations}
\nomenclature{$SHAP$}{SHapley Additive exPlanations}